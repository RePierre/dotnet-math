% Created 2020-03-27 Fri 11:02
% Intended LaTeX compiler: pdflatex
\documentclass[presentation]{beamer}
\usepackage[utf8]{inputenc}
\usepackage[T1]{fontenc}
\usepackage{graphicx}
\usepackage{grffile}
\usepackage{longtable}
\usepackage{wrapfig}
\usepackage{rotating}
\usepackage[normalem]{ulem}
\usepackage{amsmath}
\usepackage{textcomp}
\usepackage{amssymb}
\usepackage{capt-of}
\usepackage{hyperref}
\RequirePackage{fancyvrb}
\DefineVerbatimEnvironment{verbatim}{Verbatim}{fontsize=\scriptsize}
\usetheme{metropolis}
\usecolortheme{}
\usefonttheme{}
\useinnertheme{}
\useoutertheme{}
\author{Petru Rebeja, Marius Apetrii}
\date{26 Martie 2020}
\title{Tehnici Avansate de Programare}
\subtitle{Delegates,  Attributes, Lambda Expressionis \& Extension Methods}
\institute[UAIC]{Facultatea de Matematică\\Universitatea Alexandru Ioan Cuza, Iași}
\hypersetup{
 pdfauthor={Petru Rebeja, Marius Apetrii},
 pdftitle={Tehnici Avansate de Programare},
 pdfkeywords={},
 pdfsubject={},
 pdfcreator={Emacs 26.3 (Org mode 9.3.4)},
 pdflang={Romanian}}
\begin{document}

\maketitle
\section{Introducere}
\label{sec:org226baf2}
\begin{frame}[label={sec:orgfdae5e9},fragile]{Recapitulare}
 \begin{itemize}
\item Principiile \texttt{SOLID}
\item Modularizarea codului-sursă
\end{itemize}
\end{frame}
\begin{frame}[label={sec:orga8a5637}]{Despre ce vom discuta azi}
\begin{itemize}
\item Delegates
\item Expresii lambda
\item Atribute
\item Extension methods
\end{itemize}
\end{frame}
\section{\texttt{Delegates}}
\label{sec:orgbebd2b0}
\begin{frame}[label={sec:org2b1d8b5},fragile]{Ce este un \texttt{Delegate}}
 \begin{itemize}
\item Un obiect care știe să apeleze o metodă\footnote{Joseph Albahari and Ben Albahari. 2012. C\# 5.0 in a Nutshell: The Definitive Reference (5th. ed.). O’Reilly Media, Inc.\label{orgf882306}}
\item Diferențiem între:
\begin{itemize}
\item Un \texttt{tip Delegate}
\item O \texttt{instanță Delegate}
\end{itemize}
\end{itemize}
\end{frame}
\begin{frame}[label={sec:org09c9221},fragile]{Tipuri \texttt{Delegate}}
 \begin{itemize}
\item Un tip \texttt{Delegate} definește semnătura metodelor pe care le poate apela o instanță a acestui tip.
\end{itemize}
\begin{block}{Cum definim un tip Delegate:}
\begin{verbatim}
delegate TResult Transformer<TItem, TResult>(TItem item);
\end{verbatim}
\end{block}
\end{frame}
\begin{frame}[label={sec:org2d366d2},fragile]{Instanță \texttt{Delegate}}
 \begin{itemize}
\item Este \alert{slab echivalentă} cu un \texttt{pointer la o funcție} din limbajul \texttt{C}.
\item Este delegată de apelant cu invocarea propriu-zisă: apelantul invocă instanța tipului \texttt{Delegate} iar aceasta la rândul ei apelează metoda(ele) țintă\textsuperscript{\ref{orgf882306}}.
\item Decuplează apelantul de metoda-țintă.
\end{itemize}
\begin{block}{Exemplu}
\begin{verbatim}
var parse = new Transformer<string, int>(System.Convert.ToInt32);
var result = parse("2020");
\end{verbatim}
\end{block}
\end{frame}
\begin{frame}[label={sec:org1378b4b},fragile]{Instanță \texttt{Delegate}}
 Compilatorul ne oferă o sintaxă mai simplă pentru declararea instanțelor \texttt{Delegate}:
\begin{block}{Exemplu}
\begin{verbatim}
Transformer<string, int> parse = System.Convert.ToInt32;
var result = parse("2020");
\end{verbatim}
\end{block}
\end{frame}
\begin{frame}[label={sec:org319c284},fragile]{\texttt{Delegate} vs interfață\textsuperscript{\ref{orgf882306}}}
 \begin{itemize}
\item Problemele pe care le rezolvă un \texttt{Delegate} pot fi rezolvate și cu o interfață.
\item Alegem să folosim un \texttt{Delegate} atunci când:
\begin{itemize}
\item Definirea și implementarea unei interfețe necesită să scriem mult \emph{cod de umplutură}.
\item Trebuie să apelăm mai multe metode (\texttt{multicasting}).
\end{itemize}
\end{itemize}
\end{frame}
\begin{frame}[label={sec:org88e04d4},fragile]{\texttt{Multicast Delegates}}
 \begin{itemize}
\item O instanță \texttt{Delegate} poate referenția mai multe metode.
\item Acestea sunt apelate în ordinea în care au fost adăugate la instanță.
\item Adăugarea unei metode la lista de invocare se face cu operatorul \texttt{+=}.
\item Eliminarea din listă se face cu \texttt{-=}.
\end{itemize}
\end{frame}
\begin{frame}[label={sec:orgf083cff},fragile]{\texttt{Multicast Delegates}}
 \begin{block}{Exemplu}
\begin{verbatim}
delegate void ReportProgress(int percentage, string status);

void ReportProgressToConsole(int percentage, string status){
    Console.WriteLine("Progress: {0}%", percentage);
    Console.WriteLine("Status: {0}.", status);
}

void ReportProgressToFile(int percentage, string status){
    System.IO.File.WriteAllText("status.txt",
	String.Format("{0}% Done. Status: {1}.", percentage, status));
}

static void Main(){
    var reportProgress = ReportProgressToConsole;
    reportProgress+=ReportProgressToFile;
    reportProgress(50, "Waiting...");
}
\end{verbatim}
\end{block}
\end{frame}
\section{Expresii Lambda}
\label{sec:orge92fd7d}
\begin{frame}[label={sec:org5230b1f},fragile]{Ce este o expresie lambda}
 \begin{itemize}
\item O metodă fără nume menită să înlocuiască o instanță \texttt{delegate}.
\item O sintaxă mai simplă pentru declararea unei instanțe \texttt{delegate} ad-hoc.
\vskip 0.3in
\end{itemize}
\begin{block}{Forma expresiei lambda}
\vskip 0.1in
\texttt{(parametri) => expresie | \{ bloc; \}}
\end{block}
\end{frame}
\begin{frame}[label={sec:orgfbe6e05},fragile]{Expresii lambda}
 \begin{block}{Exemplu}
\begin{verbatim}
delegate void ReportProgress(int percentage, string status);

static void Main(){
    ReportProgress p = (percentage, status)=>
	Console.WriteLine("{0}%. Status: {1}.", percentage, status)
}
\end{verbatim}
\end{block}
\end{frame}
\section{Atribute}
\label{sec:org4bd8df7}
\begin{frame}[label={sec:org4a232ad},fragile]{Ce sunt atributele}
 \begin{itemize}
\item Un mecanism pentru adăugarea de date suplimentare la elementele codului-sursă (metode, clase, parametri etc.).
\item O clasă derivată din clasa abstractă \texttt{System.Attribute}.
\end{itemize}
\end{frame}
\begin{frame}[label={sec:org243bc7b},fragile]{Exemplu}
 \begin{block}{Marcarea unui tip perimat}
\begin{verbatim}
[Obsolete]
public class GodObject
{
}
\end{verbatim}
Compilatorul va emite un mesaj de avertizare la întâlnirea unei referințe către tipul \texttt{GodObject}.
\end{block}
\end{frame}
\begin{frame}[label={sec:org4f0ae51},fragile]{Alte exemple}
 \begin{itemize}
\item Serializare: atributele descriu relația de corespondență dintre un membru al clasei și un element XML,
\item Securitate: atributele conțin cerințele necesare pentru ca apelantul să aibă acces la resursa decorată,
\item Depanare: compilatorul ignoră metodele decorate cu \texttt{ConditionalAttribute}.
\end{itemize}
\end{frame}
\begin{frame}[label={sec:orga460804},fragile]{Parametrii atributelor}
 \begin{itemize}
\item Atributele pot avea două tipuri de parametri:
\begin{enumerate}
\item Parametri de \alert{ordine} (\texttt{positional}) --- sunt parametrii constructorilor publici ai atributului,
\item Parametri cu \alert{nume} (\texttt{named}) --- corespund câmpurilor și proprietăților publice ale atributului.
\end{enumerate}
\end{itemize}
\end{frame}
\begin{frame}[label={sec:org75ef5a7},fragile]{Parametrii atributelor --- exemplu}
 \begin{verbatim}
public class XmlElementAttribute: Attribute
{
    public XmlElementAttribute(string elementName)
    {
	ElementName = elementName;
    }

    public string Namespace {get; set;}
}

[XmlElement(nameof(Student), Namespace="https://www.math.uaic.ro")]
public class Student
{
}
\end{verbatim}

\noindent\rule{\textwidth}{0.5pt}
Aici \texttt{nameof(Student)} este parametru de ordine iar \texttt{Namespace="https://www.math.uaic.ro"} este parametru cu nume.
\end{frame}
\section{Metodele de extensiune}
\label{sec:org85e8fa4}
\begin{frame}[label={sec:org996fba4},fragile]{Ce sunt \texttt{extension methods}}
 \begin{itemize}
\item Metode care extind funcționalitatea unui tip existent fără să-l modifice.
\item Metode statice definite într-o clasă statică dar care sunt apelate ca și metode ale unei instanțe.
\item Primul parametru al metodei are modificatorul \texttt{this} și denotă tipul de date care va fi extins.
\end{itemize}
\end{frame}
\begin{frame}[label={sec:orgd55f517},fragile]{Exemplu}
 \begin{verbatim}
public static class DateTimeExtensions
{
    public static string ToIsoShortDate(this DateTime date)
    {
	return date.ToString("yyyy-MM-dd");
    }
}

static void Main()
{
    Console.WriteLine(DateTime.Today.ToIsoShortDate());
    // 2020-03-26
}
\end{verbatim}
\end{frame}
\begin{frame}[label={sec:org798cb1b},fragile]{Avantajele metodelor de extensiune}
 \begin{itemize}
\item Îmbunătățesc lizibilitatea codului și cresc calitatea acestuia,
\item Permit înlănțuirea apelurilor \texttt{Where().Select()}
\item Decuplează codul-sursă.
\end{itemize}
\end{frame}
\section{Încheiere}
\label{sec:org9c31a1c}
\begin{frame}[label={sec:orgfe906d3}]{Recapitulare}
\begin{itemize}
\item Delegates
\item Expresii lambda
\item Atribute
\item Extension methods
\end{itemize}
\end{frame}
\begin{frame}[label={sec:orga2bfec1}]{Vă mulțumesc!}
\begin{center}
Mulțumesc pentru atenție!
\end{center}
\end{frame}
\end{document}