% Created 2020-04-16 Thu 15:39
% Intended LaTeX compiler: pdflatex
\documentclass[presentation]{beamer}
\usepackage[utf8]{inputenc}
\usepackage[T1]{fontenc}
\usepackage{graphicx}
\usepackage{grffile}
\usepackage{longtable}
\usepackage{wrapfig}
\usepackage{rotating}
\usepackage[normalem]{ulem}
\usepackage{amsmath}
\usepackage{textcomp}
\usepackage{amssymb}
\usepackage{capt-of}
\usepackage{hyperref}
\RequirePackage{fancyvrb}
\DefineVerbatimEnvironment{verbatim}{Verbatim}{fontsize=\scriptsize}
\usetheme{metropolis}
\usecolortheme{}
\usefonttheme{}
\useinnertheme{}
\useoutertheme{}
\author{Petru Rebeja, Marius Apetrii}
\date{16 Aprilie 2020}
\title{Tehnici Avansate de Programare}
\subtitle{Baze de date și Gestiunea sarcinilor de lucru}
\institute[UAIC]{Facultatea de Matematică\\Universitatea Alexandru Ioan Cuza, Iași}
\hypersetup{
 pdfauthor={Petru Rebeja, Marius Apetrii},
 pdftitle={Tehnici Avansate de Programare},
 pdfkeywords={},
 pdfsubject={},
 pdfcreator={Emacs 26.3 (Org mode 9.3.6)},
 pdflang={Romanian}}
\begin{document}

\maketitle
\section{Introducere}
\label{sec:orga8173ff}
\begin{frame}[label={sec:orgbed4e3d}]{Recapitulare}
\begin{itemize}
\item \alert{Unit of Work} --- un șablon care ne permite să executăm toate modificările aferente bazei de date într-o singură tranzacție.
\item \alert{Dependency Injection} --- o modalitate de a-i da unei instanțe variabilele de care aceasta are nevoie separând astfel crearea de instanțe de utilizarea lor.
\item \alert{Test-Driven Development} --- un stil de dezvoltare software în care mai întâi se scriu testele pentru un anumit aspect iar mai apoi implementarea propriu-zisă.
\end{itemize}
\end{frame}
\begin{frame}[label={sec:orgd187ecf}]{Agenda}
\begin{itemize}
\item Baze de date
\item Istoricul schemei bazei de date relaționale
\item Gestiunea sarcinilor de lucru într-un proiect
\end{itemize}
\end{frame}
\section{Baze de date}
\label{sec:org963c770}
\begin{frame}[label={sec:org01db716},fragile]{Noțiuni de bază}
 \begin{block}{Bază de date}
\vskip 0.1in
O \texttt{bază de date} este o colecție organizată de date care sunt stocate și accesate de pe un calculator\footnote{\url{https://en.wikipedia.org/wiki/Database}}.
\end{block}
\end{frame}
\begin{frame}[label={sec:org0f17213}]{Noțiuni de bază}
\begin{block}{Sistem de Gestiune al Bazelor de Date}
\vskip 0.1in
Sistemul de Gestiune a Bazelor de Date este un sistem software care le permite utilizatorilor să definească, să creeze, să întrețină și să controleze accesul la baza de date\footnote{Connolly, Thomas M.; Begg, Carolyn E. (2014). Database Systems – A Practical Approach to Design Implementation and Management (6th ed.). Pearson. p. 64. ISBN 978-1292061184.}.
\end{block}
\end{frame}
\begin{frame}[label={sec:org3f2e50b}]{Noțiuni de bază}
\begin{block}{Schema bazei de date}
\vskip 0.1in
Schema bazei de date este structura logică a bazei de date descrisă într-un limbaj formal suportat de SGBD\footnote{\url{http://en.wikipedia.org/wiki/Database\_schema}} sau o reprezentare vizuală a acesteia\footnote{\url{https://www.techopedia.com/definition/30601/database-schema}}
\end{block}
\end{frame}
\begin{frame}[label={sec:org8345456},fragile]{Tipuri de baze de date}
 Există două tipuri de baze de date:
\begin{itemize}
\item Relaționale (\texttt{SQL}) și
\item Non-relaționale (\texttt{No-SQL})
\end{itemize}
\end{frame}
\begin{frame}[label={sec:orgde685f4}]{Baze de date relaționale}
\begin{itemize}
\item Sunt bazate pe tabele și modelează relația dintre ele,
\item Au o schemă predefinită,
\item Suportă interogări complexe,
\end{itemize}
\end{frame}
\begin{frame}[label={sec:org67dd947},fragile]{Baze de date relaționale (cont.)}
 \begin{itemize}
\item Suportă creșterea pe verticală (\texttt{vertical scaling}) prin adăugarea de memorie RAM, spațiu pe disk etc.,
\item Pun accentul pe proprietățile \texttt{ACID}:
\begin{itemize}
\item \texttt{Atomicity} --- modificări atomice,
\item \texttt{Consistency} --- impune consistența datelor
\item \texttt{Isolation} --- modificările se fac în izolare unele față de altele
\item \texttt{Durability} --- modificările sunt salvate pe disk.
\end{itemize}
\end{itemize}
\end{frame}
\begin{frame}[label={sec:org3b17f9d}]{Baze de date No-SQL}
\begin{itemize}
\item Sunt bazate pe documente, grafuri, perechi cheie-valoare etc.
\item \alert{Nu} au o schemă predefinită,
\item Au suport limitat pentru interogări complexe,
\end{itemize}
\end{frame}
\begin{frame}[label={sec:org97ef92f},fragile]{Baze de date No-SQL (cont.)}
 \begin{itemize}
\item Suportă creșterea pe orizontală (\texttt{horizontal scaling}) prin adăugarea de noduri noi,
\item Aplică teorema CAP\footnote{\url{https://en.wikipedia.org/wiki/CAP\_theorem}} --- în orice moment oferă două proprietăți din următoarele:
\begin{itemize}
\item \texttt{Consistency} --- orice scriere primește cele mai recente date sau o eroare,
\item \texttt{Availability} --- fiecare cerere primește un răspuns dar pot să nu fie cele mai recente,
\item \texttt{Partition tolerance} --- systemul continuă să funcționeze în ciuda pierderii unor mesaje.
\end{itemize}
\end{itemize}
\end{frame}
\section{Evoluția bazei de date}
\label{sec:orga0f5082}
\begin{frame}[label={sec:orgc8b64a2}]{Evoluția aplicației}
\begin{itemize}
\item Baza de date evoluează (de obicei) împreună cu aplicația,
\item Modificările bazei de date fac parte din ciclul de dezvoltare.
\end{itemize}
\end{frame}
\begin{frame}[label={sec:org6e62fd7}]{Bune practici}
\begin{itemize}
\item Schema bazei de date trebuie păstrată în sistemul de management al istoricului\footnote{\url{https://www.troyhunt.com/10-commandments-of-good-source-control/}} pentru a asigura sincronizarea între modificările aplicației și a bazei de date,
\item Aplicarea modificărilor trebuie sincronizată,
\item Altfel întregul sistem software devine inutilizabil.
\end{itemize}
\end{frame}
\section{Gestiunea sarcinilor de lucru}
\label{sec:org0dad813}
\begin{frame}[label={sec:org851ea9b},fragile]{Componente-cheie}
 Dezvoltarea unui sistem software implică existența următoarelor componente:
\begin{itemize}
\item Lista de cerințe/defecte (\texttt{product backlog}),
\item Fluxul de lucru în dezvoltare,
\item Documentația,
\item Ciclul de lansare.
\end{itemize}
\end{frame}
\begin{frame}[label={sec:orga4a94b4},fragile]{Lista de cerințe/defecte}
 \begin{block}{Product backlog}
\vskip 0.1in
Este o listă de cerințe și defecte ale sistemului software, ordonată după valoarea adăugată, risc și prioritate.
\vskip 0.1in
\begin{itemize}
\item \texttt{Product Backlog Item} = o cerință sau un defect din listă.
\end{itemize}
\end{block}
\end{frame}
\begin{frame}[label={sec:orgd2ec424},fragile]{Fluxul de lucru}
 \begin{itemize}
\item Totalitatea stărilor prin care trece o sarcină. Ex. (\texttt{Open}, \texttt{In Progress}, \texttt{Closed}).
\item Sarcinile sunt afișate pe o tablă separată în coloane cu o coloană pentru fiecare stare.
\item La trecerea dintr-o stare în alta sarcina este mutată în coloana respectivă.
\end{itemize}
\end{frame}
\begin{frame}[label={sec:orga4079d4},fragile]{Documentația}
 \begin{itemize}
\item Trebuie păstrată la minimul necesar și ajustată o dată cu modificările aplicației.
\item Trebuie păstrată în sistemul de gestiune al istoricului:
\begin{itemize}
\item Fiind aproape de cod este (mai) ușor de întreținut,
\item Trebuie să fie într-un format care să permită revizuirea modificărilor (ex: \texttt{markdown}).
\end{itemize}
\end{itemize}
\end{frame}
\begin{frame}[label={sec:org72a00f2}]{Ciclul de lansare}
\begin{itemize}
\item Poate fi periodic sau sporadic,
\item Necesită planificarea sarcinilor și modificărilor pentru fiecare versiune.
\item Sarcinile planificate sunt puse într-o listă nouă sau etichetate pentru a putea fi filtrate.
\end{itemize}
\end{frame}
\section{Demonstrații}
\label{sec:orga3055df}
\section{Încheiere}
\label{sec:orgf858f01}
\begin{frame}[label={sec:org91c8d78},fragile]{Recapitulare --- baze de date}
 \begin{itemize}
\item \alert{Baza de date} este o colecție organizată de date care pot fi manipulate prin intermediul unui SGBD.
\item \alert{SGBD} = Sistem de Gestiune al Bazelor de Date; permite manipulearea datelor și întreținerea bazelor de date.
\item \alert{Schema bazei de date} este reprezentarea structurii bazei de date și trebuie păstrată în sistemul de gestiune al istoricului alături de codul-sursă al aplicației.
\item Folosiți \texttt{Database project} din Visual Studio pentru modificarea schemei bazei de date.
\end{itemize}
\end{frame}
\begin{frame}[label={sec:org4079b4d}]{Recapitulare --- dezvoltare proiect}
\begin{itemize}
\item Păstrați o listă a sarcinilor de lucru a proiectului,
\item Definiți un flux de lucru,
\item Păstrați un minim necesar de documentație,
\item Concentrați-vă asupra sarcinilor planificate pentru versiunea nouă.
\end{itemize}
\end{frame}
\begin{frame}[label={sec:orgacfdbb2}]{Vă mulțumesc}
\begin{center}
Succes la evaluare!
\end{center}
\end{frame}
\end{document}