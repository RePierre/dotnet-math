% Created 2020-03-19 Thu 14:59
% Intended LaTeX compiler: pdflatex
\documentclass[presentation]{beamer}
\usepackage[utf8]{inputenc}
\usepackage[T1]{fontenc}
\usepackage{graphicx}
\usepackage{grffile}
\usepackage{longtable}
\usepackage{wrapfig}
\usepackage{rotating}
\usepackage[normalem]{ulem}
\usepackage{amsmath}
\usepackage{textcomp}
\usepackage{amssymb}
\usepackage{capt-of}
\usepackage{hyperref}
\RequirePackage{fancyvrb}
\DefineVerbatimEnvironment{verbatim}{Verbatim}{fontsize=\scriptsize}
\usetheme{metropolis}
\usecolortheme{}
\usefonttheme{}
\useinnertheme{}
\useoutertheme{}
\author{Petru Rebeja, Marius Apetrii}
\date{19 Martie 2020}
\title{Tehnici Avansate de Programare}
\subtitle{Modularizarea codului-sursă}
\institute[UAIC]{Facultatea de Matematică\\Universitatea Alexandru Ioan Cuza, Iași}
\hypersetup{
 pdfauthor={Petru Rebeja, Marius Apetrii},
 pdftitle={Tehnici Avansate de Programare},
 pdfkeywords={},
 pdfsubject={},
 pdfcreator={Emacs 26.3 (Org mode 9.3.4)},
 pdflang={Romanian}}
\begin{document}

\maketitle
\section{Introducere}
\label{sec:org67630a6}
\begin{frame}[label={sec:org3bc015e},fragile]{Recapitulare}
 \begin{itemize}
\item \texttt{Fluxul de lucru Git} permite să implementăm cerințe noi fără să afectăm aplicația în vreun fel.
\item Ecosistemul \texttt{NuGet} ne permite să integrăm biblioteci terțe pentru a ne concentra asupra logicii aplicației.
\item Principiile \texttt{SOLID} ne ajută să scriem cod de calitate ridicată.
\end{itemize}
\end{frame}
\begin{frame}[label={sec:org698ceaf}]{Despre ce vom discuta azi}
\begin{itemize}
\item Cum organizăm codul-sursă în soluție
\end{itemize}
\end{frame}
\section{Concepte de bază în Visual Studio}
\label{sec:org61661c9}
\begin{frame}[label={sec:org120f5b3}]{Soluție}
\begin{itemize}
\item O mulțime de resurse necesare pentru crearea unei aplicații.
\item Resursele sunt grupate (de obicei) în proiecte.
\end{itemize}
\end{frame}
\begin{frame}[label={sec:orga97867a},fragile]{Proiect}
 \begin{itemize}
\item O submulțime de resurse menite să fie compilate împreună în:
\begin{itemize}
\item O bibliotecă software (\texttt{.dll}),
\item O aplicație executabilă (\texttt{.exe}),
\item O aplicație Web.
\end{itemize}
\end{itemize}
\end{frame}
\begin{frame}[label={sec:org21b3d10},fragile]{Spațiu de nume (\texttt{namespace})}
 \begin{itemize}
\item Denotă un domeniu restrâns căruia îi aparține o mulțime de tipuri.
\item Este folosit pentru a diferenția tipurile cu același nume.
\end{itemize}
\end{frame}
\begin{frame}[label={sec:org5a00851},fragile]{Denumirea spațiilor de nume}
 Șablonul recomandat pentru denumirea spațiilor de nume noi este\footnote{\url{https://docs.microsoft.com/en-us/dotnet/standard/design-guidelines/names-of-namespaces}}:
\begin{verbatim}
<Company>.(<Product>|<Technology>)[.<Feature>][.<Subnamespace>]
\end{verbatim}
\end{frame}
\section{Modularizarea codului sursă}
\label{sec:org62bfdf1}
\begin{frame}[label={sec:org0fabc35}]{Recapitulare: Acuplarea}
\begin{quotation} %% Acuplarea
\alert{Acuplarea} este o măsură a gradului de interdependență dintre modulele unui produs software\footnote{\url{https://en.wikipedia.org/wiki/Coupling\_(computer\_programming)}}.
\end{quotation}
\end{frame}
\begin{frame}[label={sec:org77342e3}]{Recapitulare: Coeziunea}
\begin{quotation} %% Coeziunea
\alert{Coeziunea} este măsura în care elementele unui modul aparțin unul de celălalt\footnote{\url{https://en.wikipedia.org/wiki/Cohesion\_(computer\_science)}}.
\end{quotation}
\end{frame}
\section{De ce modularizăm codul-sursă?}
\label{sec:orgd5e70ad}
\begin{frame}[label={sec:org75ebbc4}]{Separarea funcționalităților}
\begin{itemize}
\item Fiecare modul este responsabil de o singură funcționalitate.
\item Gradul ridicat de coeziune și gradul redus de acuplare rezultă în cod ușor de întreținut.
\end{itemize}
\end{frame}
\begin{frame}[label={sec:orgd08bcc9},fragile]{Reutilizarea codului-sursă}
 \begin{itemize}
\item Modulele pot fi combinate pentru a obține aplicații noi (ex: pachete \texttt{NuGet}).
\item Aceleași module pot fi reutilizate pentru publicarea aplicației pe diferite platforme.
\end{itemize}
\end{frame}
\begin{frame}[label={sec:org11c0295}]{Încapsulare}
\begin{itemize}
\item Modulul expune doar interfața publică și ascunde detaliile de implementare.
\item Ulterior detaliile de implementare pot fi modificate fără să afecteze funcționalitatea consumatorilor.
\end{itemize}
\end{frame}
\begin{frame}[label={sec:org8ff58c1}]{Ușor de testat}
\begin{itemize}
\item Este mai ușor să scrii teste pentru module cu acuplare mică și coeziune mare.
\item La rândul lor și testele devin suficient de mici și modulare.
\end{itemize}
\end{frame}
\section{Încheiere}
\label{sec:orge06ba92}
\begin{frame}[label={sec:orgfcbf27d}]{Recapitulare}
Modularizarea codului-sursă ne permite să:
\begin{itemize}
\item Separăm funcționalitățile/responsabilitățile,
\item Încapsulăm și reutilizăm codul-sursă și
\item Să testăm mai codul mai ușor/rapid.
\end{itemize}
\end{frame}
\begin{frame}[label={sec:org4211c7f}]{Vă mulțumesc!}
\begin{center}
Mulțumesc pentru atenție!
\end{center}
\end{frame}
\end{document}