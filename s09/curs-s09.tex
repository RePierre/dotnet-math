% Created 2021-04-15 Thu 07:11
% Intended LaTeX compiler: pdflatex
\documentclass[presentation]{beamer}
\usepackage[utf8]{inputenc}
\usepackage[T1]{fontenc}
\usepackage{graphicx}
\usepackage{grffile}
\usepackage{longtable}
\usepackage{wrapfig}
\usepackage{rotating}
\usepackage[normalem]{ulem}
\usepackage{amsmath}
\usepackage{textcomp}
\usepackage{amssymb}
\usepackage{capt-of}
\usepackage{hyperref}
\RequirePackage{fancyvrb}
\DefineVerbatimEnvironment{verbatim}{Verbatim}{fontsize=\scriptsize}
\usetheme{metropolis}
\usecolortheme{}
\usefonttheme{}
\useinnertheme{}
\useoutertheme{}
\author{Petru Rebeja, Marius Apetrii}
\date{15 Aprilie 2021}
\title{Tehnici Avansate de Programare}
\subtitle{\texttt{Test-Driven Development} și Recapitulare}
\institute[UAIC]{Facultatea de Matematică\\Universitatea Alexandru Ioan Cuza, Iași}
\hypersetup{
 pdfauthor={Petru Rebeja, Marius Apetrii},
 pdftitle={Tehnici Avansate de Programare},
 pdfkeywords={},
 pdfsubject={},
 pdfcreator={Emacs 26.3 (Org mode 9.4.5)},
 pdflang={Romanian}}
\begin{document}

\maketitle
\section{Introducere}
\label{sec:org04ccc11}
\begin{frame}[label={sec:org92c969e}]{Recapitulare}
\begin{itemize}
\item \alert{Unit of Work} --- un șablon care ne permite să executăm toate modificările aferente bazei de date într-o singură tranzacție.
\item \alert{Dependency Injection} --- o modalitate de a-i da unei instanțe variabilele de care aceasta are nevoie separând astfel crearea de instanțe de utilizarea lor.
\end{itemize}
\end{frame}
\begin{frame}[label={sec:org608c10d},fragile]{Agenda}
 \begin{itemize}
\item \texttt{Test-Driven Development}
\item Recapitulare
\end{itemize}
\end{frame}
\section{Test-Driven Development}
\label{sec:org6bca3ba}
\begin{frame}[label={sec:org97005cb}]{Definiție}
\begin{block}{Test-Driven Development\footnote{\url{https://www.agilealliance.org/glossary/tdd}}}
Este un stil de dezvoltare a aplicațiilor constând în trecerea frecventă prin următoarele trei activități:
\begin{itemize}
\item Scrierea de cod
\item Scrierea de teste unitare și
\item Refactorizare.
\end{itemize}
\end{block}
\end{frame}
\begin{frame}[label={sec:orgf9e52b7},fragile]{Ciclul de dezvoltare \texttt{TDD}\footnote{\url{https://www.agilealliance.org/glossary/tdd}}}
 \begin{itemize}
\item Scierea unui singur test unitar menit să descrie un anumit aspect al metodei,
\item Execuția testului care eșuează deoarece aspectul încă nu este implementat,
\item Scrierea cantității de cod necesare pentru a trece testul,
\item Refactorizarea codului scris pentru a se încadra în diverse criterii (simplitate, lizibilitate etc.).
\end{itemize}
\end{frame}
\begin{frame}[label={sec:org94c78e5},fragile]{Ciclul de dezvoltare \texttt{TDD}}
 Procesul se repetă de mai multe ori ducând la acumularea de teste unitare pentru fiecare aspect al aplicației.
\end{frame}
\begin{frame}[label={sec:org23aaa28}]{Avantaje}
\begin{itemize}
\item Reduce timpul petrecut pentru depanare,
\item Reduce numărul defectelor din aplicație,
\item Scade gradul de acuplare și crește coeziunea.
\end{itemize}
\end{frame}
\begin{frame}[label={sec:org6d3e55e}]{Dezavantaje}
\begin{itemize}
\item Crește timpul necesar pentru dezvoltare,
\item Crește efortul necesar întreținerii,
\item Necesită o perioadă de adaptare.
\end{itemize}
\end{frame}
\begin{frame}[label={sec:orgfd52fc0},fragile]{Teste unitare}
 \begin{itemize}
\item Metode care verifică un anumit aspect al modulului pe care îl testează.
\item Sunt decorate cu un anumit atribut (ex.: \texttt{[TestMethod]}).
\item Clasa care le conține poate fi și ea decorată cu un anumit atribut (ex.: \texttt{[TestClass]}).
\end{itemize}
\end{frame}
\begin{frame}[label={sec:org8d98d5e},fragile]{Structura unui test}
 Un test este structurat conform șablonului \texttt{AAA}:
\begin{itemize}
\item \alert{Arrange}  --- pregătirile necesare pentru a apela metoda testată: crearea de instanțe false, instanțierea clasei etc.
\item \alert{Act} --- apelarea propriu-zisă a metodei testate și preluarea rezultatelor.
\item \alert{Assert} --- verificarea rezultatelor și/sau comportamentului metodei testate.
\end{itemize}
\end{frame}
\section{Demonstrații}
\label{sec:org8152035}
\section{Încheiere}
\label{sec:org3c2490c}
\begin{frame}[label={sec:org7f48693},fragile]{Recapitulare --- \texttt{TDD}}
 \begin{itemize}
\item \alert{Test-Driven Development} --- un stil de dezvoltare software în care mai întâi se scriu testele pentru un anumit aspect iar mai apoi implementarea propriu-zisă.
\end{itemize}
\end{frame}
\begin{frame}[label={sec:org2d9e188},fragile]{Recapituralre --- \texttt{SOLID}}
 \begin{description}
\item[{S}] Single Responsibility Principle
\item[{O}] Open-Closed Principle
\item[{L}] Liskov Substitution Principle
\item[{I}] Interface Segregation Principle
\item[{D}] Dependency Inversion Principle
\end{description}
\end{frame}
\begin{frame}[label={sec:org4171031},fragile]{Recapitulare --- \texttt{extension methods}}
 \begin{itemize}
\item Metode statice care extind funcționalitatea unui tip existent fără să-l modifice.
\end{itemize}
\end{frame}
\begin{frame}[label={sec:org6d79396},fragile]{Recapitulare --- \texttt{Repository} \& \texttt{UnitOfWork}}
 \begin{itemize}
\item \alert{Repository Pattern} --- șablon de proiectare ce decuplează logica aplicației de accesul la date.
\item \alert{Unit of Work} --- un șablon care ne permite să executăm toate modificările aferente bazei de date într-o singură tranzacție.
\end{itemize}
\end{frame}
\begin{frame}[label={sec:org351c364}]{Vă mulțumesc}
\begin{center}
Succes la evaluare!
\end{center}
\end{frame}
\end{document}