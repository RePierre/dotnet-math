% Created 2020-04-02 Thu 15:24
% Intended LaTeX compiler: pdflatex
\documentclass[presentation]{beamer}
\usepackage[utf8]{inputenc}
\usepackage[T1]{fontenc}
\usepackage{graphicx}
\usepackage{grffile}
\usepackage{longtable}
\usepackage{wrapfig}
\usepackage{rotating}
\usepackage[normalem]{ulem}
\usepackage{amsmath}
\usepackage{textcomp}
\usepackage{amssymb}
\usepackage{capt-of}
\usepackage{hyperref}
\RequirePackage{fancyvrb}
\DefineVerbatimEnvironment{verbatim}{Verbatim}{fontsize=\scriptsize}
\usetheme{metropolis}
\usecolortheme{}
\usefonttheme{}
\useinnertheme{}
\useoutertheme{}
\author{Petru Rebeja, Marius Apetrii}
\date{2 Aprilie 2020}
\title{Tehnici Avansate de Programare}
\subtitle{LINQ, Entity Framework \& Repository Pattern}
\institute[UAIC]{Facultatea de Matematică\\Universitatea Alexandru Ioan Cuza, Iași}
\hypersetup{
 pdfauthor={Petru Rebeja, Marius Apetrii},
 pdftitle={Tehnici Avansate de Programare},
 pdfkeywords={},
 pdfsubject={},
 pdfcreator={Emacs 26.3 (Org mode 9.3.4)},
 pdflang={Romanian}}
\begin{document}

\maketitle
\section{Introducere}
\label{sec:org764d22f}
\begin{frame}[label={sec:org7a980c1}]{Recapitulare}
\begin{itemize}
\item Delegates
\item Expresii lambda
\item Atribute
\item Extension methods
\end{itemize}
\end{frame}
\begin{frame}[label={sec:org1203f17}]{Despre ce vom discuta azi}
\begin{itemize}
\item LINQ
\item Entity Framework
\item Repository Pattern
\end{itemize}
\end{frame}
\section{LINQ}
\label{sec:orgc5ed647}
\begin{frame}[label={sec:org22fd0fd},fragile]{Ce este LINQ}
 \begin{itemize}
\item LINQ = Language Integrated Query.
\item O mulțime de funcții ale platformei \texttt{.net} și particularități ale limbajului de programare care permit interogarea uniformă atât a colecțiilor de date din memorie cât și a surselor de date externe\footnote{Joseph Albahari and Ben Albahari. 2012. C\# 5.0 in a Nutshell: The Definitive Reference (5th. ed.). O’Reilly Media, Inc.\label{org03724bb}}.
\end{itemize}
\end{frame}
\begin{frame}[label={sec:org5851f0e},fragile]{Sintaxa LINQ}
 Interogările LINQ pot fi scrise folosind două tipuri de sintaxă:
\begin{itemize}
\item Sintaxa de interogare (\texttt{Query syntax}),
\item Sintaxa pe bază de înlănțuire de funcții (\texttt{Fluent syntax}).
\end{itemize}
\end{frame}
\begin{frame}[label={sec:org1aaba56},fragile]{Query syntax}
 \begin{itemize}
\item O sintaxă simplificată.
\item \alert{Nu} este o transpunere a sintaxei \texttt{SQL} în \texttt{C\#}.
\item A fost inspirată în mare parte de limbajele funcționale (\texttt{LISP}, \texttt{Haskel}); \texttt{SQL} a avut o influență mai mult cosmetică\textsuperscript{\ref{org03724bb}}.
\end{itemize}
\end{frame}
\begin{frame}[label={sec:org16e5b60},fragile]{Query syntax --- exemplu}
 \begin{verbatim}
var numbers = new int[]{1, 2, 3, 4};

var odd = from n in numbers
	  where n % 2 == 1
	  orderby n descending
	  select n;
\end{verbatim}
\end{frame}
\begin{frame}[label={sec:org675d16c}]{Fluent syntax}
\begin{itemize}
\item Sintaxa fundamentală și cea mai flexibilă.
\item Reprezintă o înlănțuire a operatorilor (metode de extensiune care operează pe colecții de elemente).
\end{itemize}
\end{frame}
\begin{frame}[label={sec:org7861756},fragile]{Fluent syntax --- exemplu}
 \begin{verbatim}
var numbers = new int[]{1, 2, 3, 4};

var odd = numbers
    .Where(n => n % 2 == 1)
    .OrderByDescending(n => n);
\end{verbatim}
\end{frame}
\begin{frame}[label={sec:org0d32471},fragile]{Operatori standard}
 Platforma \texttt{.net} pune la dispoziție o mulțime de \alert{operatori LINQ standard} în clasa \href{https://docs.microsoft.com/en-us/dotnet/api/system.linq.enumerable?view=netcore-3.1}{\texttt{Enumerable}}.


Exemple:
\begin{itemize}
\item \texttt{Where()}
\item \texttt{Select()}
\item \texttt{Average()}
\item \texttt{GroupBy()}
\item etc.
\end{itemize}
\end{frame}
\begin{frame}[label={sec:orgf84a667},fragile]{Execuție întârziată}
 \begin{itemize}
\item O interogare \texttt{LINQ} este executată abia atunci când începem să parcurgem mulțimea-rezultat.
\item Excepții:
\begin{itemize}
\item Operatorii care au ca rezultat un scalar: \texttt{First()}, \texttt{Count()} etc.,
\item Operatorii care convertesc într-o colecție: \texttt{ToList()} etc.
\end{itemize}
\end{itemize}
\end{frame}
\begin{frame}[label={sec:org6c2a884},fragile]{Execuție întârziată --- exemplu\textsuperscript{\ref{org03724bb}}}
 \begin{verbatim}
var numbers = new List<int>(){1};
var multiples = numbers.Select(n => n * 10);
numbers.Add(2);

foreach(var n in multiples)
{
    Console.Write($"{n}|");
}
\end{verbatim}
\begin{block}{Rezultat}
\texttt{10|20|}
\end{block}
\end{frame}
\section{Entity Framework}
\label{sec:org017daf4}
\begin{frame}[label={sec:org5173c99},fragile]{Ce este un ORM}
 \begin{block}{\texttt{Object-Relational Mapping}}
Este o tehnică prin care interogarea și manipularea datelor dintr-o bază de date se face folosind paradigma orientat-obiect\footnote{\url{https://stackoverflow.com/a/1279678/844006}}.
\end{block}
\end{frame}
\begin{frame}[label={sec:orgdc52189},fragile]{Entity Framework}
 \begin{block}{Entity Framework}
Este o bibliotecă pentru platforma \texttt{.net} care oferă funcționalitatea \texttt{ORM}.
\end{block}
\end{frame}
\begin{frame}[label={sec:org47dc205},fragile]{Fluxuri de lucru}
 \texttt{Entity Framework} oferă două moduri de a modela baza de date:
\begin{enumerate}
\item Din cod --- programatorul definește asocierile dintre clase și tabele precum și relațiile dintre diferite tabele.
\item Folosind interfața grafică --- utilizatorul definește schema bazei de date cu ajutorul unei interfețe grafice iar codul este generat de platformă.
\end{enumerate}
\end{frame}
\begin{frame}[label={sec:orgf0faf21},fragile]{Alegerea fluxului de lucru\footnote{\url{https://docs.microsoft.com/en-us/ef/ef6/modeling/\#ef-workflows}}}
 \begin{center}
\begin{tabular}{lll}
 & Prefer să scriu cod & Prefer interfață grafică\\
\hline
Bază de date nouă & \texttt{Code First} & \texttt{Model First}\\
Bază de date existentă & \texttt{Code First} & \texttt{Database First}\\
\end{tabular}
\end{center}
\end{frame}
\begin{frame}[label={sec:orgbd587db}]{Definirea modelului}
\begin{block}{Modelare}
Este procedeul prin care definim tipurile de obiecte și relațiile dintre ele a.î. să reprezinte procesele din viața reală pe care le simulează aplicația curentă.
\end{block}
\end{frame}
\begin{frame}[label={sec:org2039d22}]{Definirea modelului}
\begin{block}{Modelul bazei de date}
Totalitatea claselor și altor tipuri de date care reprezintă structura bazei de date a aplicației.
\end{block}
\end{frame}
\begin{frame}[label={sec:orge6d5fa1},fragile]{Noțiuni de bază}
 \begin{block}{Data context / \texttt{DbContext}}
\begin{itemize}
\item Reprezintă o sesiune pentru interogarea și manipularea datelor din baza de date.
\item Interfață pentru acces la date și proprietăți conexe (conexiune, tranzacții etc.).
\end{itemize}
\end{block}
\end{frame}
\begin{frame}[label={sec:org9af0cf3}]{Noțiuni de bază}
\begin{block}{Entity}
\begin{itemize}
\item Denumire generică pentru fiecare tip de date din modelul bazei de date.
\end{itemize}
\end{block}
\end{frame}
\begin{frame}[label={sec:orge0349fc},fragile]{Noțiuni de bază}
 \begin{block}{Relație / \texttt{Navigational Properity}}
\begin{itemize}
\item O proprietate a unei entități prin care se modelează relația cu altă entitate,
\item Ex: \texttt{Student.Classes} este o listă ce modelează relația \texttt{1:N} dintre entitățile \texttt{Student} și \texttt{Class}.
\end{itemize}
\end{block}
\end{frame}
\begin{frame}[label={sec:org9f71e33}]{Noțiuni de bază}
\begin{block}{CRUD}
\begin{itemize}
\item Un acronim care denotă cele 4 funcții ale unui model de date: \alert{Create}, \alert{Read}, \alert{Update}, \alert{Delete}.
\end{itemize}
\end{block}
\end{frame}
\begin{frame}[label={sec:org14f6ced},fragile]{Eager vs Lazy loading}
 \begin{itemize}
\item Implicit Entity Framework va încărca entitățile relaționate atunci când acestea sunt parcurse (\texttt{Lazy loading}).
\item Acest comportament poate fi modificat a.î. entitățile relaționate să fie încărcate în același timp cu entitatea principală (\texttt{Eager loading}).
\end{itemize}
\end{frame}
\begin{frame}[label={sec:orgce053a3},fragile]{Eager vs Lazy loading}
 \begin{itemize}
\item \texttt{Lazy loading} scade presiunea asupra bazei de date dar necesită o conexiune activă la parcurgere,
\item \texttt{Eager loading} --- invers.
\end{itemize}
\end{frame}
\section{Repository Pattern}
\label{sec:org8da0e41}
\begin{frame}[label={sec:orgbd6cf53},fragile]{Ce este \texttt{Repository}}
 \begin{itemize}
\item Șablon de proiectare care decuplează logica aplicației de accesul la date.
\item Interfață pentru a implementa operațiile \texttt{CRUD} pentru o anumită entitate.
\end{itemize}
\end{frame}
\begin{frame}[label={sec:org5a4b266}]{Avantaje}
\begin{itemize}
\item Decuplează logica aplicației de accesul la date,
\item Decuplează aplicația de modificările aplicate bazei de date,
\item Promovează reutilizarea codului,
\item Logica aplicației devine mai ușor de testat.
\end{itemize}
\end{frame}
\begin{frame}[label={sec:org270044a},fragile]{Exemplu}
 \begin{verbatim}
public interface IStudentRepository
{
    Student GetById(int studentId);
    void Update(Student student);
    void Insert(Student student);
    void Delete(int studentId);
    IQueryable<Student> Query();
    void Save();
}
\end{verbatim}
\end{frame}
\section{Încheiere}
\label{sec:org16cf87e}
\begin{frame}[label={sec:org729baf2}]{Recapitulare}
\pause
\begin{block}{LINQ}
\pause
O mulțime de funcții ce permit interogarea uniformă a colecțiilor de date din diverse surse.
\pause
\end{block}
\begin{block}{Entity Framework}
\pause
O platformă care permite interogarea și manipularea datelor folosind paradigma orientat-obiect.
\pause
\end{block}
\begin{block}{Repository Pattern}
\pause
Șablon de proiectare ce decuplează logica aplicației de accesul la date.
\end{block}
\end{frame}
\begin{frame}[label={sec:orgcdbd6ed}]{Vă mulțumesc!}
\begin{center}
Mulțumesc pentru atenție!
\end{center}
\end{frame}
\end{document}