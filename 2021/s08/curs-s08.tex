% Created 2021-04-08 Thu 06:44
% Intended LaTeX compiler: pdflatex
\documentclass[presentation]{beamer}
\usepackage[utf8]{inputenc}
\usepackage[T1]{fontenc}
\usepackage{graphicx}
\usepackage{grffile}
\usepackage{longtable}
\usepackage{wrapfig}
\usepackage{rotating}
\usepackage[normalem]{ulem}
\usepackage{amsmath}
\usepackage{textcomp}
\usepackage{amssymb}
\usepackage{capt-of}
\usepackage{hyperref}
\RequirePackage{fancyvrb}
\DefineVerbatimEnvironment{verbatim}{Verbatim}{fontsize=\scriptsize}
\usetheme{metropolis}
\usecolortheme{}
\usefonttheme{}
\useinnertheme{}
\useoutertheme{}
\author{Petru Rebeja, Marius Apetrii}
\date{8 Aprilie 2021}
\title{Tehnici Avansate de Programare}
\subtitle{Unit of Work pattern \& Dependency Injection}
\institute[UAIC]{Facultatea de Matematică\\Universitatea Alexandru Ioan Cuza, Iași}
\hypersetup{
 pdfauthor={Petru Rebeja, Marius Apetrii},
 pdftitle={Tehnici Avansate de Programare},
 pdfkeywords={},
 pdfsubject={},
 pdfcreator={Emacs 26.3 (Org mode 9.4.4)},
 pdflang={Romanian}}
\begin{document}

\maketitle
\section{Introducere}
\label{sec:orgf39e77a}
\begin{frame}[label={sec:org0f08fa3}]{Recapitulare}
\begin{itemize}
\item \alert{LINQ} --- O mulțime de funcții ce permit interogarea uniformă a colecțiilor de date din diverse surse.
\item \alert{Entity Framework} --- O platformă care permite interogarea și manipularea datelor folosind paradigma orientat-obiect.
\item \alert{Repository Pattern} --- Șablon de proiectare ce decuplează logica aplicației de accesul la date.
\end{itemize}
\end{frame}
\begin{frame}[label={sec:orgb2b6948}]{Agenda}
\begin{itemize}
\item Unit of Work pattern
\item Dependency Injection
\end{itemize}
\end{frame}
\section{\texttt{Unit of Work} pattern}
\label{sec:org3a889e6}
\begin{frame}[label={sec:org081f1eb},fragile]{Repository}
 \begin{verbatim}
public interface IRepository<TEntity>
{
    IQueryable<TEntity> Query();
    void Insert(TEntity entity);
    void Update(TEntity entity);
    void Delete(TEntity entityId);
    void SaveChanges();
}
\end{verbatim}
\end{frame}
\begin{frame}[label={sec:org87894b6},fragile]{Repository}
 \begin{itemize}
\item Folosim o instanță de \texttt{repository} pentru a efectua operații \texttt{CRUD} asupra uneia sau a mai multor entități de același tip.
\item Toate modificările sunt păstrate în memorie.
\item Metoda \texttt{SaveChanges()} este apelată pentru a salva modificările în baza de date.
\end{itemize}
\end{frame}
\begin{frame}[label={sec:org8f3ae94},fragile]{Scenariu problematic}
 Vrem să introducem în baza de date un client nou și datele de contact ale acestuia:

\begin{verbatim}
customerRepository.Insert(new Customer{...});
customerRepository.SaveChanges();

addressRepository.Insert(new CustomerAddress{...});
addressRepository.SaveChanges();
\end{verbatim}
\pause
Ce se întâmplă dacă \texttt{addressRepository.SaveChanges()} aruncă o excepție?
\end{frame}
\begin{frame}[label={sec:org9ce1c8f},fragile]{Scenariu problematic}
 Ce se întâmplă dacă \texttt{addressRepository.SaveChanges()} aruncă o excepție?
\begin{itemize}
\item Clientul este salvat în baza de date dar adresa nu.
\item Baza de date rămâne într-o stare neconsistentă.
\item Pentru a reveni la consistență clientul fără adresă trebuie șters.
\end{itemize}
\end{frame}
\begin{frame}[label={sec:orgbe52551}]{Scenariu problematic}
\begin{itemize}
\item Astfel de scenarii se regăsesc des în cerințele aplicațiilor.
\item Atunci când lucrăm cu mai mult de două entități asigurarea consistenței datelor devine un mecanism foarte complex.
\end{itemize}
\end{frame}
\begin{frame}[label={sec:org0e737a0},fragile]{Soluția}
 Pentru a evita astfel de probleme folosim \texttt{Unit of Work}.
\end{frame}
\begin{frame}[label={sec:org52a49bf},fragile]{Unit of Work}
 \begin{block}{Unit of Work}
\texttt{Unit of Work} ține evidența tuturor modificărilor aferente bazei de date efectuate în timpul unei tranzacții logice din aplicație. La finalul tranzacției modificările sunt salvate în baza de date într-un mod care asigură atomicitatea și integritatea datelor.\footnote{\url{https://www.martinfowler.com/eaaCatalog/unitOfWork.html}}
\end{block}
\end{frame}
\begin{frame}[label={sec:org62ff8df},fragile]{Unit of Work Pattern}
 \begin{verbatim}
public interface IRepository<TEntity>
{
    IQueryable<TEntity> Query();
    void Insert(TEntity entity);
    void Update(TEntity entity);
    void Delete(TEntity entityId);
}

public interface IUnitOfWork
{
    void Commit();
}
\end{verbatim}
\begin{itemize}
\item Metoda \texttt{SaveChanges()} devine redundantă și trebuie eliminată din interfața \texttt{IRepository}.
\end{itemize}
\end{frame}
\begin{frame}[label={sec:orga10523e},fragile]{Exemplu}
 \begin{verbatim}
public class RegisterCustomerTransaction
{
    public RegisterCustomerTransaction(
	IRepository<Customer> customerRepository,
	IRepository<CustomerAddress> addressRepository,
	IUnitOfWork unitOfWork)
    {
	...
    }

    public void Execute(Customer customer, CustomerAddress address)
    {
	customerRepository.Insert(customer);
	addressRepository.Insert(address);
	unitOfWork.Commit();
    }
}
\end{verbatim}
\end{frame}
\section{\texttt{Depencency Injection}}
\label{sec:org5e8228d}
\begin{frame}[label={sec:org34d646c},fragile]{Definiție}
 \begin{block}{Dependency Injection}
\texttt{Dependency Injection} este o tehnică prin care un obiect îi furnizează altuia dependențele de care acesta din urmă are nevoie.\footnote{\url{https://en.wikipedia.org/wiki/Dependency\_injection}}
\end{block}
\end{frame}
\begin{frame}[label={sec:orgdf59dae},fragile]{Definiție}
 \begin{block}{Dependency Injection}
\texttt{Dependency Injection} este un termen de 25 dolari pentru un concept de 5 cenți. [\ldots{}] \texttt{Dependency Injection} înseamnă să-i dai unei instanțe a unui anumit obiect variabilele de care are nevoie.\footnote{\url{https://www.jamesshore.com/Blog/Dependency-Injection-Demystified.html}}
\end{block}
\end{frame}
\begin{frame}[label={sec:org5d64f76}]{Avantaje}
\begin{itemize}
\item Separarea responsabilităților\footnote{\url{https://en.wikipedia.org/wiki/Dependency\_injection}}:
\begin{itemize}
\item \alert{Injectorul} construiește dependențele și le injectează în client și
\item \alert{Clientul} manipulează dependențele pentru obținerea unui anumit scop.
\end{itemize}
\item Reducerea codului de umplutură și a gradului de acuplare.
\end{itemize}
\end{frame}
\begin{frame}[label={sec:orgbb25372}]{Avantaje}
\begin{itemize}
\item Flexibilitatea la schimbări fără necesitatea de recompilare --- efectele execuției codului client pot fi modificate prin injectarea unei implementări diferite.
\item Facilitează testarea în izolare.
\end{itemize}
\end{frame}
\begin{frame}[label={sec:org6813d9f},fragile]{Dezavantaje}
 \begin{itemize}
\item Grad de abstractizare mai mare care crește dificultatea codului.
\item Induce o dependență față de o bibliotecă pentru \texttt{Dependency Injection}.
\item Poate duce la o acuplare mai mare prin utilizarea defectuoasă.
\end{itemize}
\end{frame}
\begin{frame}[label={sec:orgfdd4dea},fragile]{\texttt{Service Locator}}
 \begin{block}{\texttt{Service Locator}}
  Ideea de bază a unui \texttt{Service Locator} este să ai la dispoziție un obiect care știe cum să obțină, la cerere, instanțe ale tuturor serviciilor necesare unei aplicații\footnote{\url{https://martinfowler.com/articles/injection.html\#UsingAServiceLocator}}.
\vskip 0.5in
\alert{N.B.}: O instanță de serviciu = o dependență.
\end{block}
\end{frame}
\begin{frame}[label={sec:orgc68b1a5},fragile]{\texttt{Service Locator} --- exemplu}
 \begin{verbatim}
public class RegisterCustomerTransaction
{
    public RegisterCustomerTransaction()
    {
	customerRepository = ServiceLocator
	    .GetInstance<IRepository<Customer>>();
	addressRepository = ServiceLocator
	    .GetInstance<IRepository<CustomerAddress>>();
	unitOfWork = ServiceLocator.GetInstance<IUnitOfWork>();
    }
}
\end{verbatim}
\end{frame}
\begin{frame}[label={sec:org2d060d1},fragile]{\texttt{Service Locator}}
 \begin{itemize}
\item \alert{Este un anti-șablon}\footnote{\url{https://blog.ploeh.dk/2010/02/03/ServiceLocatorisanAnti-Pattern/}}.
\item Induce un grad mare de acuplare.
\item Ascunde dependențele.
\end{itemize}
\end{frame}
\section{Încheiere}
\label{sec:org455a509}
\begin{frame}[label={sec:org2ed473f}]{Recapitulare}
\pause
\begin{block}{Unit of Work}
\pause
Este un șablon care ne permite să executăm toate modificările aferente bazei de date într-o singură tranzacție.
\pause
\end{block}
\begin{block}{Dependency Injection}
\pause
Este o modalitate de a-i da unei instanțe variabilele de care aceasta are nevoie separând astfel crearea de instanțe de utilizarea lor.
\end{block}
\end{frame}
\begin{frame}[label={sec:org13f10c0}]{Vă mulțumesc!}
\begin{center}
Mulțumesc pentru atenție!
\end{center}
\end{frame}
\end{document}