% Created 2021-03-17 Wed 13:53
% Intended LaTeX compiler: pdflatex
\documentclass[presentation]{beamer}
\usepackage[utf8]{inputenc}
\usepackage[T1]{fontenc}
\usepackage{graphicx}
\usepackage{grffile}
\usepackage{longtable}
\usepackage{wrapfig}
\usepackage{rotating}
\usepackage[normalem]{ulem}
\usepackage{amsmath}
\usepackage{textcomp}
\usepackage{amssymb}
\usepackage{capt-of}
\usepackage{hyperref}
\RequirePackage{fancyvrb}
\DefineVerbatimEnvironment{verbatim}{Verbatim}{fontsize=\scriptsize}
\usetheme{metropolis}
\usecolortheme{}
\usefonttheme{}
\useinnertheme{}
\useoutertheme{}
\author{Petru Rebeja, Marius Apetrii}
\date{18 Martie 2021}
\title{Tehnici Avansate de Programare}
\subtitle{Principiile \texttt{SOLID} și modularizarea codului-sursă}
\institute[UAIC]{Facultatea de Matematică\\Universitatea Alexandru Ioan Cuza, Iași}
\hypersetup{
 pdfauthor={Petru Rebeja, Marius Apetrii},
 pdftitle={Tehnici Avansate de Programare},
 pdfkeywords={},
 pdfsubject={},
 pdfcreator={Emacs 26.3 (Org mode 9.4.4)},
 pdflang={Romanian}}
\begin{document}

\maketitle
\section{Introducere}
\label{sec:org2428c5f}
\begin{frame}[label={sec:org5e734f2},fragile]{Recapitulare}
 \begin{itemize}
\item Resurse gestionate automat vs resurse gestionate manual,
\item Organizarea în 3 generații,
\item Gestionarea ineficientă a resurselor are impact negativ,
\item Folosim \texttt{NuGet} pentru adăugarea bibliotecilor terțe.
\end{itemize}
\end{frame}
\begin{frame}[label={sec:org3f24f36},fragile]{Despre ce vom discuta azi}
 \begin{itemize}
\item Principiile \texttt{SOLID}
\item Cum organizăm codul-sursă în soluție
\end{itemize}
\end{frame}
\section{Principiile \texttt{SOLID}}
\label{sec:org23a6d92}
\begin{frame}[label={sec:org90a5503}]{SOLID}
\begin{description}
\item[{S}] Single Responsibility Principle
\item[{O}] Open-Closed Principle
\item[{L}] Liskov Substitution Principle
\item[{I}] Interface Segregation Principle
\item[{D}] Dependency Inversion Principle
\end{description}
\end{frame}
\begin{frame}[label={sec:orgca02fb6},fragile]{Context}
 Vrem să simulăm tranzacții financiare modelând diverse tipuri de conturi bancare (\texttt{Debit}, \texttt{Credit} etc.) precum și operațiunile care se fac asupra acestora.
\end{frame}
\begin{frame}[label={sec:org30d808a}]{Single Responsibility Principle}
\begin{block}{Single Responsibility Principle}
Fiecare modul trebuie să fie responsabil pentru un singur aspect legat de funcționalitatea oferită de sistemul software. Mai mult, acel aspect trebuie să fie încapsulat în întregime de modulul responsabil\footnote{\url{https://en.wikipedia.org/wiki/Single\_responsibility\_principle}}.
\end{block}
\end{frame}
\begin{frame}[label={sec:orgb8ccf3f}]{Open-Closed Principle}
\begin{block}{Open-Closed Principle}
Părțile componente ale unui sistem software trebuie să fie ușor de extins dar greu de modificat\footnote{\url{https://en.wikipedia.org/wiki/Open/closed\_principle}}.
\end{block}
\end{frame}
\begin{frame}[label={sec:org18988f0}]{Liskov Substitution Principle}
\begin{block}{Liskov Substitution Principle}
Obiectele unui sistem software trebuie să fie substituibile de către instanțe ale unor subtipuri de obiecte, fără ca substituția să afecteze corectitudinea sistemului\footnote{\url{https://en.wikipedia.org/wiki/Liskov\_substitution\_principle}}.
\end{block}
\end{frame}
\begin{frame}[label={sec:orgdb7cb3f}]{Interface Segregation Principle}
\begin{block}{Interface Segregation Principle}
Interfețele trebuie să fie mici și specifice contextului de utilizare; nu mari și generale\footnote{\url{https://en.wikipedia.org/wiki/Interface\_segregation\_principle}}.
\end{block}
\end{frame}
\begin{frame}[label={sec:org1269d28}]{Dependency Inversion Principle}
\begin{block}{Dependency Inversion Principle}
Modulele unui sistem software trebuie să depindă de reprezentări abstracte și nu de implementări concrete\footnote{\url{https://en.wikipedia.org/wiki/Dependency\_inversion\_principle}}.
\end{block}
\end{frame}
\section{Concepte de bază în Visual Studio}
\label{sec:orge6bc502}
\begin{frame}[label={sec:orga67f53c}]{Soluție}
\begin{itemize}
\item O mulțime de resurse necesare pentru crearea unei aplicații.
\item Resursele sunt grupate (de obicei) în proiecte.
\end{itemize}
\end{frame}
\begin{frame}[label={sec:org98cb3eb},fragile]{Proiect}
 \begin{itemize}
\item O submulțime de resurse menite să fie compilate împreună în:
\begin{itemize}
\item O bibliotecă software (\texttt{.dll}),
\item O aplicație executabilă (\texttt{.exe}),
\item O aplicație Web.
\end{itemize}
\end{itemize}
\end{frame}
\begin{frame}[label={sec:org63daafe},fragile]{Spațiu de nume (\texttt{namespace})}
 \begin{itemize}
\item Denotă un domeniu restrâns căruia îi aparține o mulțime de tipuri.
\item Este folosit pentru a diferenția tipurile cu același nume.
\end{itemize}
\end{frame}
\begin{frame}[label={sec:org55572f9},fragile]{Denumirea spațiilor de nume}
 Șablonul recomandat pentru denumirea spațiilor de nume noi este\footnote{\url{https://docs.microsoft.com/en-us/dotnet/standard/design-guidelines/names-of-namespaces}}:
\begin{verbatim}
<Company>.(<Product>|<Technology>)[.<Feature>][.<Subnamespace>]
\end{verbatim}
\end{frame}
\section{Modularizarea codului sursă}
\label{sec:org84a2016}
\begin{frame}[label={sec:org9569cd9}]{Recapitulare: Acuplarea}
\begin{quotation} %% Acuplarea
\alert{Acuplarea} este o măsură a gradului de interdependență dintre modulele unui produs software\footnote{\url{https://en.wikipedia.org/wiki/Coupling\_(computer\_programming)}}.
\end{quotation}
\end{frame}
\begin{frame}[label={sec:orga78ca0c}]{Recapitulare: Coeziunea}
\begin{quotation} %% Coeziunea
\alert{Coeziunea} este măsura în care elementele unui modul aparțin unul de celălalt\footnote{\url{https://en.wikipedia.org/wiki/Cohesion\_(computer\_science)}}.
\end{quotation}
\end{frame}
\section{De ce modularizăm codul-sursă?}
\label{sec:orgf7b30f3}
\begin{frame}[label={sec:orgdb072d7}]{Separarea funcționalităților}
\begin{itemize}
\item Fiecare modul este responsabil de o singură funcționalitate.
\item Gradul ridicat de coeziune și gradul redus de acuplare rezultă în cod ușor de întreținut.
\end{itemize}
\end{frame}
\begin{frame}[label={sec:orga3b644c},fragile]{Reutilizarea codului-sursă}
 \begin{itemize}
\item Modulele pot fi combinate pentru a obține aplicații noi (ex: pachete \texttt{NuGet}).
\item Aceleași module pot fi reutilizate pentru publicarea aplicației pe diferite platforme.
\end{itemize}
\end{frame}
\begin{frame}[label={sec:org2ca9c6b}]{Încapsulare}
\begin{itemize}
\item Modulul expune doar interfața publică și ascunde detaliile de implementare.
\item Ulterior detaliile de implementare pot fi modificate fără să afecteze funcționalitatea consumatorilor.
\end{itemize}
\end{frame}
\begin{frame}[label={sec:org6694515}]{Ușor de testat}
\begin{itemize}
\item Este mai ușor să scrii teste pentru module cu acuplare mică și coeziune mare.
\item La rândul lor și testele devin suficient de mici și modulare.
\end{itemize}
\end{frame}
\section{Încheiere}
\label{sec:org39a69f9}
\begin{frame}[label={sec:orgd64be01},fragile]{Recapitulare --- \texttt{SOLID}}
 \begin{itemize}
\item \alert{Single Responsibility Principle}
\item \alert{Open-Closed Principle}
\item \alert{Liskov Substitution Principle}
\item \alert{Interface Segregation Principle}
\item \alert{Dependency Inversion Principle}
\end{itemize}
\end{frame}
\begin{frame}[label={sec:org0585e6f}]{Recapitulare --- modularizare}
Modularizarea codului-sursă ne permite să:
\begin{itemize}
\item Separăm funcționalitățile/responsabilitățile,
\item Încapsulăm și reutilizăm codul-sursă și
\item Să testăm mai codul mai ușor/rapid.
\end{itemize}
\end{frame}
\begin{frame}[label={sec:org3f6ed04}]{Vă mulțumesc!}
\begin{center}
Mulțumesc pentru atenție!
\end{center}
\end{frame}
\end{document}